\documentclass{article}
\usepackage{amsfonts}
\usepackage{amsmath}
\usepackage{mathtools}
\allowdisplaybreaks
\DeclareMathOperator\E{\mathbb{E}}
\DeclareMathOperator\Var{Var}
\pagestyle{empty}
\setlength\parindent{0pt}
\begin{document}
We start with \(\hat{Y} = \alpha + \beta X\).
The parametrization is a little artificial: \(\alpha\) is the intercept, but maybe \(X\) is never \(0\), so it doesn't make much sense to think of it as `the value \(\hat{Y}\) takes when \(X = 0\).'
`More natural units' for \(\hat{Y}\) and \(X\) would be their differences from the means.
\begin{align*}
\hat{Y} &= \alpha + \beta X\\
(\hat{Y} - \E Y) + \E Y&= \alpha + \beta (X - \E X) + \beta \E X\\
(\hat{Y} - \E Y) &= \alpha + \beta (X - \E X) + \beta \E X - \E Y\\
(\hat{Y} - \E Y) &= \big[\alpha - (\E Y - \beta \E X)\big] + \beta (X - \E X)
\end{align*}
So we can interpet \(\alpha = \E Y - \beta \E X\) as `if \(X\) and \(Y\) both had mean zero, we'd want \(\hat{Y} = 0\) when \(X = 0\).'
\end{document}
